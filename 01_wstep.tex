\section{Wstęp}

\begin{frame}{Czym są modele LLM?}
    \begin{block}{Definicja}
        \textbf{LLM (Large Language Model)} to model uczenia maszynowego, który potrafi generować, tłumaczyć i przetwarzać tekst, bazując na przewidywaniu kolejnych słów w sekwencji.
    \end{block}

    \vspace{0.5cm}
    Główne zastosowania:
    \begin{itemize}
        \item Generowanie kodu programistycznego.
        \item Tłumaczenia maszynowe (np. Google Translate, DeepL).
        \item Chatboty i asystenci (np. ChatGPT, Gemini).
        \item Analiza sentymentu.
    \end{itemize}
\end{frame}

\begin{frame}{Ewolucja modeli językowych}
    Krótka historia rozwoju NLP (Natural Language Processing):
    \begin{enumerate}
        \item \textbf{Bag-of-Words} – proste zliczanie słów.
        \item \textbf{RNN (Recurrent Neural Networks)} – analiza sekwencyjna.
        \item \textbf{LSTM} – pamięć długoterminowa w sieciach.
        \item \textbf{Transformer (2017)} – przełom dzięki mechanizmowi \textit{Self-Attention}.
    \end{enumerate}
\end{frame}