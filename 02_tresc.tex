\section{Rozdział główny}

\begin{frame}{Główna treść - definicje}
    \begin{block}{Ważna definicja}
        Podział na pliki (ang. \textit{splitting}) pozwala utrzymać porządek w kodzie źródłowym, zwłaszcza gdy prezentacja ma kilkadziesiąt slajdów.
    \end{block}
\end{frame}

\begin{frame}{Główna treść - wyliczenia}
    Przykładowa lista numerowana w głównym rozdziale:
    \begin{enumerate}
        \item Punkt pierwszy.
        \item Punkt drugi.
        \item Punkt trzeci.
    \end{enumerate}
    
    \vspace{0.5cm}
    Przykładowe równanie:
    \begin{equation}
        a^2 + b^2 = c^2
    \end{equation}
\end{frame}